% ----------------------------------------------------------
% Introdução 
% Capítulo sem numeração, mas presente no Sumário
% ----------------------------------------------------------

\chapter[Introdução]{Introdução}
% \addcontentsline{toc}{chapter}{Introdução}

A osteoartrite (OA), popularmente conhecida como artrose, é uma forma muito comum de doença reumática, caracterizada como uma condição multifatorial e degenerativa que afeta desde a cartilagem articular até os ossos adjacentes, resultando em sintomas de dor, deformidade e perda de função \cite{Kraus2015, PACCA2018}. Esses impactos comprometem significativamente a qualidade de vida, especialmente em grupos mais afetados, como idosos, mulheres e indivíduos obesos \cite{PACCA2018}. Além de sua alta prevalência, a OA é uma das principais causas de incapacidade no mundo, com maior incidência na articulação do joelho, seguida pelo quadril e pela mão. Dados de 2020 apontam que a doença afeta cerca de 7,6\% da população global, e projeções indicam um aumento de 60 a 100\% até 2050 \cite{COURTIES20241397}.

Exercícios de propriocepção e fortalecimento muscular, assim como terapias farmacêuticas, têm sido aplicados a pacientes diagnosticados com OA de joelho com o objetivo de controlar ou reduzir os sintomas de dor, uma vez que não existem medicamentos capazes de retardar o seu desenvolvimento \cite{Sardim2020, Lin2009}. Essa abordagem é especialmente apropriada para pacientes em estágios iniciais da doença, quando a cartilagem ainda não foi completamente degradada \cite{Kanamoto2020}. No entanto, o diagnóstico depende da experiência e do julgamento clínico do profissional na interpretação das radiografias, o que pode levar a inconsistências entre o grau previsto e o grau real, devido às mínimas diferenças entre os estágios adjacentes da doença \cite{KELLGREN1957, Mohammed2023}. Esses desafios têm impulsionado estudos sobre sistemas automáticos de detecção e classificação da OA de joelho.

A introdução de técnicas de inteligência artificial (IA) nos últimos anos tem permitido a automação de tarefas que antes eram realizadas manualmente, incluindo a interpretação de imagens médicas \cite{WANG2024103201}. Alguns exemplos incluem a detecção de pneumonia \citeonline{9077899}, a identificação e classificação de câncer de pulmão em tomografias computadorizadas e a detecção de retinopatia diabética em imagens de fundo de olho \cite{8697352, Dai2021}.

No campo da reumatologia, a visão computacional também tem sido aplicada à detecção de OA de joelho a partir de radiografias, com o objetivo de automatizar o processo de diagnóstico, reduzir a subjetividade da interpretação humana e realizar a classificação da severidade da doença através da escala de Kellgren/Lawrence (KL) \cite{Mohammed2023}. Esses estudos têm se concentrado em utilizar arquiteturas de aprendizado profundo, como redes neurais convolucionais (RNCs), e compará-las entre si para identificar qual abordagem oferece melhor desempenho na classificação da severidade da OA. No entanto, a operação de convolução limita o relacionamento entre pixels distantes em uma imagem, o que pode prejudicar a capacidade de captar dependências de longo alcance em radiografias \cite{Shamshad2023}.

Como uma abordagem alternativa, ou até complementar, foram propostas arquiteturas baseadas em transformers, capazes de apresentar um excelente desempenho em tarefas de classificação, como é o caso do vision transformer (ViT) \cite{Dosovitskiy2021}. Essas arquiteturas têm sido aplicadas com sucesso em tarefas relacionadas à medicina, como o diagnóstico de COVID-19 a partir de radiografias, classificação de tumores e doenças de retina, tornando-se o estado da arte nesta área \cite{Shamshad2023}.

Este trabalho propõe, portanto, uma análise comparativa entre o desempenho de modelos de RNC e ViT na tarefa de detecção e classificação da OA de joelho seguindo a escala KL a partir de radiografias. A avaliação considera métricas gerais de desempenho, eficiência computacional, incerteza preditiva e interpretabilidade, com o objetivo de identificar qual abordagem é mais adequada para uso como uma ferramenta auxiliar em diagnósticos clínicos. Para isso, foram utilizadas técnicas de pré-processamento de imagens, seleção dos melhores hiperparâmetros e estratégias de treinamento, bem como a avaliação dos modelos de classificação propostos.

\section{Objetivos}

\subsection{Objetivo Geral}

O objetivo geral deste trabalho consistiu em realizar uma análise comparativa completa entre modelos de RNC e ViT na tarefa de detectar e classificar a OA de joelho usando radiografias, facilitando assim o diagnóstico da doença por meio de uma ferramenta automatizada.

\subsection{Objetivos Específicos}

\begin{itemize}
    \item Realizar uma revisão bibliográfica sobre a OA de joelho e as técnicas de visão computacional aplicadas à detecção de doenças reumáticas;
    \item Treinar os modelos propostos para classificar a severidade da OA de joelho a partir de um conjunto de dados público;
    \item Comparar os modelos de RNC e ViT com base em métricas de performance, eficiência computacional, incerteza preditiva e interpretabilidade;
    \item Analisar os resultados obtidos e discutir as vantagens e desvantagens de cada abordagem.
\end{itemize}

A metodologia proposta para atingir os objetivos deste trabalho consistiu nas seguintes etapas: coleta e pré-processamento de um conjunto de dados de radiografias de joelhos com diferentes graus de severidade da OA seguindo a escala KL; implementação da \textit{pipeline} de treinamento dos modelos para classificar a severidade da OA de joelho com hiperparâmetros fixos; avaliação dos modelos com base em métricas de performance, tempos de treinamento e inferência; aplicação da predição conformal para análise quantitativa; interpretação visual dos mapas de ativação; análise dos resultados obtidos e discussão das vantagens e desvantagens de cada abordagem.

\section{Organização do Trabalho}

Este trabalho está organizado em seis capítulos, incluindo a introdução. No \autoref{cap:fundamentacao}, são apresentados os conceitos e definições necessárias para o entendimento deste trabalho, incluindo a osteoartrite de joelho e suas características clínicas, conceitos fundamentais das arquiteturas de aprendizado profundo, incluindo as RNCs e os ViTs. No \autoref{cap:trab_relacionados}, são abordados os trabalhos relacionados. No \autoref{cap:proposta}, é apresentada a metodologia proposta para atingir os objetivos deste trabalho, assim como a avaliação dos modelos. No \autoref{cap:resultados}, são apresentados os resultados obtidos e discussões. Por fim, no \autoref{cap:conclusao}, são apresentadas as conclusões finais deste trabalho, apontando as contribuições, limitações e sugestões para trabalhos futuros.