% ----------------------------------------------------------
% Introdução 
% Capítulo sem numeração, mas presente no Sumário
% ----------------------------------------------------------

\chapter*[Introdução]{Introdução}
\addcontentsline{toc}{chapter}{Introdução}

A osteoartrite (OA) é uma forma muito comum de doença articular, definida como uma condição degenerativa que se inicia nas articulações e afeta principalmente a cartilagem, o revestimento articular e os ligamentos \cite{Sardim2020}, resultando em sintomas de dor, rigidez e mobilidade articular limitada \cite{PACCA2018}. Tais fatores podem comprometer significativamente a qualidade de vida, especialmente em idosos e indivíduos obesos \cite{Luis2022}. A OA é altamente prevalente, sendo uma das principais causas de incapacidade no mundo, com grande incidência em articulações como joelhos e quadris, afetando uma em cada sete pessoas globalmente \cite{KELLGREN1957}. De acordo com um estudo do World Health Organization (WHO), em 2023, estimava-se a prevalência global da OA de joelho em 365 milhões de indivíduos, com maior predominância em pessoas idosas e mulheres, com cerca de 70\% e 60\%, respectivamente \cite{who2023}.

Terapias farmacêuticas têm sido aplicadas a pacientes diagnosticados com OA de joelho com o objetivo de reduzir os sintomas de dor, uma vez que não existem medicamentos capazes de retardar o seu desenvolvimento. No entanto, a progressão da doença pode ser prevenida com um diagnóstico precoce, ou seja, nos estágios iniciais em que a OA de joelho ainda é reversível \cite{Kanamoto2020}. A medicina comumente avalia a severidade da doença através dos graus de Kellgren/Lawrence (KL), que categoriza a doença em cinco níveis de progressão: 0 (saudável), 1 (duvidoso), 2 (mínimo), 3 (moderado) e 4 (severo), dependendo da experiência e cuidado médico na interpretação das radiografias \cite{KELLGREN1957}. Isso pode levar a inconsistências entre o grau previsto e o grau real da OA de joelho, devido às mínimas diferenças entre os estágios adjacentes da doença \cite{Mohammed2023}. Estudos indicam que procedimentos como artroscopia são invasivos e podem causar complicações \cite{Saraev2020}, enquanto técnicas como tomografia computadorizada e ressonância magnética também são usadas, mas o diagnóstico pode ser impreciso por falta falta de experiência do profissional \cite{Alshamrani2023}. Esses desafios têm impulsionado estudos sobre sistemas automáticos de detecção e classificação da OA de joelho.

Nos últimos anos, muitas áreas têm visto a introdução de sistemas de inteligência artificial (IA) para executar tarefas que eram realizadas de forma manual, incluindo na área da medicina para o diagnóstico de patologias, por exemplo. Avanços recentes em técnicas de aprendizado de máquina no campo da saúde levaram a uma aceleração no diagnóstico de diversas doenças, incluindo a OA de joelho \cite{Mohammed2023}. O uso de modelos de aprendizado profundo baseados em redes neurais convolucionais (RNCs) tem ganhado espaço no que tange tarefas relacionadas a visão computacional \cite{Tariq2023}. Porém, isso só foi possível após a introdução de novas técnicas para treinar redes profundas em paralelo com avanços a nível de hardware \cite{Litjens2017}. Aprendizado por transferência também é amplamente utilizado para reduzir uso de recursos computacionais para tarefas que já são executadas por modelos existentes, como as redes residuais (ResNet), Visual Geometry Group (VGG) e as redes densamente conectadas (DenseNet) \cite{Tariq2023}. Enquanto o uso de RNCs tem se mostrado útil em soluções de detecção em imagens médicas, a operação de convolução limita o relacionamento entre pixels distantes numa imagem. Para tanto, a habilidade de codificar dependências de longo alcance tem sido possível graças às arquiteturas de aprendizado profundas baseadas em atenção, como o Vision Transformer (ViT). Tais modelos de ViT têm sido empregados para várias tarefas, incluindo classificação e detecção de objetos \cite{Shamshad2023}.

A relevância desta pesquisa reside na necessidade de aprimorar o processo de diagnóstico da osteoartrite de joelho, uma doença que afeta milhões de pessoas em todo o mundo e cuja detecção precoce é crucial para retardar sua progressão. O diagnóstico manual, feito por radiologistas, muitas vezes é subjetivo e suscetível a erros, o que pode levar a diagnósticos tardios ou incorretos. A aplicação de RNCs oferece uma solução promissora para automatizar esse processo, proporcionando uma avaliação mais precisa e eficiente a partir de radiografias. Essa automatização pode reduzir a carga dos profissionais de saúde e aumentar a acessibilidade de diagnósticos mais rápidos e confiáveis. Além disso, a comparação entre arquiteturas de RNCs e modelos baseados em transformers é relevante para identificar qual abordagem oferece melhor desempenho na classificação da severidade da osteoartrite.

O objetivo deste trabalho consiste em realizar uma comparação entre as métricas de performance e eficiência computacional de RNCs e modelos de ViTs na tarefa de detecção e classificação da OA de joelho seguindo a escala de Kellgren/Lawrence a partir de radiografias, com o intuito de identificar qual abordagem é mais adequada para uso em diagnósticos clínicos. Para atingir esse objetivo, será necessário estudar as modelos de RNCs e ViTs e propor uma arquitetura capaz de solucionar o problema. Em seguida, deverá ser feito o treinamento dos modelos e, por fim, será realizada uma análise detalhada das métricas de performance, incluindo acurácia, tempo de processamento e consumo de recursos, permitindo uma avaliação comparativa das duas abordagens.

A metodologia adotada envolve o uso de técnicas de pré-processamento de imagens para reduzir ruído das radiografias, melhorar contraste e ampliar o conjunto de dados para melhorar a performance e evitar o problema de \textit{overfitting}. Em seguida, diferentes arquiteturas de RNCs serão treinadas usando a estratégia de \textit{transfer learning}, e seus resutados comparados com os modelos de vision transformer treinados para a mesma tarefa.