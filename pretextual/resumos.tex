% ---
% RESUMOS
% ---

% RESUMO em português
\setlength{\absparsep}{18pt} % ajusta o espaçamento dos parágrafos do resumo
\begin{resumo}
A osteoartrite de joelho (OA) é uma das condições articulares mais comuns e incapacitantes no mundo, sendo caracterizada como uma doença progressiva que afeta principalmente a cartilagem do joelho. Embora não tenha cura, a detecção precoce é fundamental para prevenir sua progressão. A radiografia é a principal técnica utilizada para o diagnóstico da OA e para sua classificação com base na escala de Kellgren/Lawrence (KL). No entanto, o diagnóstico radiológico depende da experiência, interpretação e tempo do profissional, o que pode gerar inconsistências ou erros. Nesse contexto, técnicas de aprendizado profundo oferecem uma alternativa mais rápida e eficiente, permitindo a automação da detecção e classificação da OA de joelho. Este estudo propõe uma comparação entre modelos de redes neurais convolucionais (RNCs) e vision transformers (ViTs) na tarefa de classificar a severidade da OA de joelho, abrangendo os modelos ResNet-34, ResNet-50, ResNet-101, VGG-16, VGG-19, DenseNet-121, DenseNet-169, Inception-v3, DeiT, Swin Transformer, DaViT, MaxViT e GC ViT. A análise comparativa considera tanto métricas de performance, após o uso de \textit{transfer learning}, quanto o consumo computacional envolvido no treinamento dos modelos. Após a realização dos experimentos, observou-se que as arquiteturas ResNet-50 e DenseNet-169 obtiveram os melhores desempenhos, com acurácias de 72,48\% e 73,19\% na classificação da OA de joelho em cinco classes, respectivamente.

 \textbf{Palavras-chaves}: Classificação. osteoartrite de joelho. radiografias. redes neurais convolucionais. transfer-learning. vision transformers.
\end{resumo}

% ABSTRACT in english
\begin{resumo}[Abstract]
 \begin{otherlanguage*}{english}
  Knee osteoarthritis (OA) is one of the most common and debilitating joint conditions worldwide, characterized as a progressive disease that primarily affects the knee cartilage. Although there is no cure, early detection is crucial to prevent its progression. Radiography is the main technique used to diagnose OA and classify it based on the Kellgren/Lawrence (KL) scale. However, radiological diagnosis depends on the professional's experience, interpretation, and time, which can lead to inconsistencies or errors. In this context, deep learning techniques offer a faster and more efficient alternative, enabling the automation of knee OA detection and classification. This study proposes a comparison between convolutional neural network (CNN) models and vision transformers (ViTs) for the task of classifying knee OA severity, including the models ResNet34, ResNet50, ResNet101, VGG16, VGG19, DenseNet121, DenseNet169, Inception, ViT-B/16, DeiT, Swin Transformer, and ResNet50-ViT-B/16. The comparative analysis considers both performance metrics, following the application of transfer learning, and the computational resources required to train the models. It is expected that the dense networks (DenseNet121 and DenseNet169), along with the hybrid architecture ResNet50-ViT-B/16, will get the best results.

   \vspace{\onelineskip}
 
   \noindent 
   \textbf{Keywords}: Classification. convolutional neural networks. knee osteoarthritis. radiographs. transfer-learning. vision transformers.
 \end{otherlanguage*}
\end{resumo}