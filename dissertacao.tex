% ---------------------------------------------------------------------------
% ---------------------------------------------------------------------------
% Junção de templates encontrados no Overleaf
% facilitando a vida do aluno de pós da UFABC
% Modelo LaTex para preparação do documento final de Dissertação de Mestrado
% ninguém do programa de pós da informação validou se presta
% ---------------------------------------------------------------------------
% ---------------------------------------------------------------------------

\documentclass[
	% -- opções da classe memoir --
	12pt,					% tamanho da fonte
	openright,				% capítulos começam em pág ímpar (insere página vazia caso preciso)
	twoside,					% para impressão em verso e anverso. Oposto a oneside
	a4paper,					% tamanho do papel. 
	% -- opções da classe abntex2 --
	%chapter=TITLE,			% títulos de capítulos convertidos em letras maiúsculas
	% section=TITLE,			% títulos de seções convertidos em letras maiúsculas
	% subsection=TITLE,		% títulos de subseções convertidos em letras maiúsculas
	% subsubsection=TITLE,	% títulos de subsubseções convertidos em letras maiúsculas
	% -- opções do pacote babel --
	english,					% idioma adicional para hifenização
	%french,					% idioma adicional para hifenização
	%spanish,				% idioma adicional para hifenização
	brazil,					% o último idioma é o principal do documento
	]{abntex2}

% ---------------------
% Pacotes OBRIGATÓRIOS
% ---------------------
%\usepackage{newtxtext,newtxmath}	% Usa a fonte Times New Roman
%\usepackage{lmodern}				% Usa a fonte Latin Modern			
\usepackage[T1]{fontenc}			% Selecao de codigos de fonte.
\usepackage[utf8]{inputenc}		% Codificacao do documento (conversão automática dos acentos)
\usepackage{lastpage}			% Usado pela Ficha catalográfica
\usepackage{indentfirst}			% Indenta o primeiro parágrafo de cada seção.
\usepackage{color}				% Controle das cores
\usepackage{graphicx,graphicx}	% Inclusão de gráficos
\usepackage{epsfig,subfig}		% Inclusão de figuras
\usepackage{microtype} 			% Melhorias de justificação
% ---------------------

\usepackage{graphicx}
\usepackage{array}

% ---------------------
% Pacotes ADICIONAIS
% ---------------------
\usepackage{lipsum}						% Geração de dummy text
\usepackage{amsmath,amssymb,mathrsfs}	% Comandos matemáticos avançados 
% \usepackage{setspace}  					% Para permitir espaçamento simples, 1 1/2 e duplo
\usepackage{verbatim}					% Para poder usar o ambiente "comment"
\usepackage{tabularx} 					% Para poder ter tabelas com colunas de largura auto-ajustável
\usepackage{afterpage} 					% Para executar um comando depois do fim da página corrente
\usepackage{url} 						% Para formatar URLs (endereços da Web)
\usepackage{multirow}				    % Para juntar duas linhas em uma só	
% ---------------------

% ---------------------
% Pacotes de CITAÇÕES
% ---------------------
\usepackage[brazilian,hyperpageref]{backref}	% Paginas com as citações na bibl
% \usepackage[alf]{abntex2cite}				% Citações padrão ABNT (alfa)
% \usepackage[num]{abntex2cite}				% Citações padrão ABNT (numericas)
% \usepackage[style=authoryear]{biblatex} % Citações padrão ABNT (numericas)
\usepackage[authoryear,round]{natbib}
\bibliographystyle{plainnat}
% ---------------------

\usepackage{rotating}

% Configurações de CITAÇÕES para abntex2
% \include{extras/conf_citacoes}

% Inclusão de dados para CAPA e FOLHA DE ROSTO (título, autor, orientador, etc.)
% ---
% Informações de dados para CAPA e FOLHA DE ROSTO
% ---
\titulo{Detecção e Classificação Automática de Osteoartrite de Joelho em Radiografias Utilizando Visão Computacional}
\autor{Guilherme de Sousa Santos}
\local{Santo André - SP}
\data{15 de agosto de 2025}
\orientador{Hugo Puertas de Araújo}
% \coorientador{Fulano Nome do Coorientador}
\instituicao{%
  Universidade Federal do ABC -- UFABC
  \par
  Centro de Matemática, Computação e Cognição 
  \par
  Bacharelado em Ciência da Computação}
\tipotrabalho{Projeto de Graduação}
% O preambulo deve conter o tipo do trabalho, o objetivo,
% o nome da instituição e a área de concentração
\preambulo{\textbf{Projeto de Graduação} apresentado ao Programa de Graduação em Ciência da Computação (área de concentração: Visão Computacional), como parte dos requisitos necessários para a obtenção do título de Bacharel em Ciência da Computação.}
% ---

% Inclui Configurações de aparência do PDF Final
\include{extras/conf_pdf}

% O tamanho da identação do parágrafo é dado por:
\setlength{\parindent}{1.3cm}

% Controle do espaçamento entre um parágrafo e outro:
\setlength{\parskip}{0.2cm}  % tente também \onelineskip

% Altera as fontes dos títulos e legendas para a família serifada (padrão do corpo de texto)
\renewcommand{\ABNTEXchapterfont}{\rmfamily}
\renewcommand{\ABNTEXpartfont}{\rmfamily}
\renewcommand{\ABNTEXsectionfont}{\rmfamily}
\renewcommand{\ABNTEXsubsectionfont}{\rmfamily}
\renewcommand{\ABNTEXsubsubsectionfont}{\rmfamily}
\renewcommand{\captionnamefont}{\rmfamily\bfseries}
\renewcommand{\captiontitlefont}{\rmfamily}

% ---------------------
% Compila o indice
% ---------------------
\makeindex
% ---------------------

\newtheorem{theorem}{Teorema}
\newcommand{\deit}{⚗}

\renewcommand{\ABNTEXsectionfont}{\rmfamily\bfseries\small}
% \renewcommand{\ABNTEXchapterfont}{\rmfamily\bfseries\small}

%%%%%%%%%%%%%%%%%%%%%%%%%%%
%%  INICIO DO DOCUMENTO  %%
%%%%%%%%%%%%%%%%%%%%%%%%%%%
\begin{document}

% Retira espaço extra obsoleto entre as frases.
\frenchspacing

% ----------------------------------------------------------
% ELEMENTOS PRÉ-TEXTUAIS (Capa, Resumo, Abstract, etc.)
% ----------------------------------------------------------
\pretextual

% Capa
\include{pretextual/capa}

% Folha de rosto (o * indica que haverá a ficha bibliográfica)
\imprimirfolhaderosto*

% Imprimir Ficha Catalografica
% \include{pretextual/catalografica}

% Inserir Folha de Aprovação
% \include{pretextual/assinaturas}

% Dedicatória
% % ---
% Dedicatória
% ---
\begin{dedicatoria}
	\vspace*{\fill}
	\begin{flushright}
		\textit{Dedico este trabalho à minha família e amigos, \\
		pelo apoio incondicional e amor \\
		em todas as etapas da minha vida acadêmica.}
	\end{flushright}
\end{dedicatoria}
% ---

% Agradecimentos
% % ---
% Agradecimentos
% ---
\begin{agradecimentos}

Agradeço primeiramente a Deus, pela saúde, força e determinação para que eu pudesse concluir esta etapa muito importante. Aos meus pais, por todo amor e paciência nos últimos anos de graduação, além de todo apoio e exemplo de vida. Ao meu orientador, Prof. Dr. Hugo Puertas, por todos os conselhos, direcionamentos e ajuda durante todo o desenvolvimento deste trabalho. Aos colegas e amigos que levo para a vida, pela parceria, motivação e momentos de descontração que tornaram esta jornada mais leve. À Universidade Federal do ABC (UFABC), pelo suporte, excelência e infraestrutura disponibilizados. E a todos que, de forma direta ou indireta, contribuíram para a realização deste trabalho, meu muito obrigado.

\end{agradecimentos}
%% ---

% Epígrafe
% \include{pretextual/epigrafe}

% Resumo e Abstract
% ---
% RESUMOS
% ---

% RESUMO em português
\setlength{\absparsep}{18pt} % ajusta o espaçamento dos parágrafos do resumo
\begin{resumo}
  A osteoartrite (OA) de joelho é uma das condições articulares mais comuns e incapacitantes no mundo, sendo caracterizada como uma doença progressiva que afeta principalmente a cartilagem do joelho. Embora não tenha cura, a detecção precoce é fundamental para prevenir sua progressão, e a radiografia é a principal técnica utilizada para o diagnóstico da OA e para sua classificação com base na escala de Kellgren/Lawrence (KL). No entanto, o diagnóstico radiológico depende da experiência, interpretação e tempo do profissional, o que pode gerar inconsistências ou erros. Nesse contexto, técnicas de aprendizado profundo oferecem uma alternativa mais rápida e eficiente, permitindo a automação da detecção e classificação da OA de joelho.

  Este estudo propõe uma comparação entre modelos de redes neurais convolucionais (RNCs) e vision transformers (ViTs) na tarefa de classificar a severidade da OA de joelho, abrangendo os modelos ResNet-34, ResNet-50, ResNet-101, VGG-16, VGG-19, DenseNet-121, DenseNet-169, Inception-v3, DeiT, Swin Transformer, DaViT, MaxViT e GCViT. O treinamento dos modelos foi realizado com o uso de aprendizado por transferência, e a análise comparativa considera métricas de performance, consumo computacional, análise quantitativa de incerteza e interpretabilidade. Os resultados mostraram que as arquiteturas RNCs, especialmente aquelas da família DenseNet apresentaram o melhor desempenho geral, com o modelo DenseNet-169 alcançando uma acurácia de 78,85\%. Em termos de eficiência computacional, as RNCs foram significativamente mais rápidas, com o DenseNet-121 oferecendo o melhor equilíbrio entre alto desempenho preditivo (QWK de 0,8878) e baixo custo de treinamento e inferência (3,11 ms/imagem). Os ViTs, apesar de competitivos, apresentaram um desempenho inferior com um custo computacional maior. Finalmente, a análise de interpretabilidade com Grad-CAM confirmou que os modelos de melhor desempenho baseiam suas decisões em marcadores patológicos relevantes, como o espaço articular e osteófitos.

  \vspace{\onelineskip}

 \textbf{Palavras-chaves}: Classificação. osteoartrite de joelho. radiografias. redes neurais convolucionais. transfer-learning. vision transformers.
\end{resumo}

% ABSTRACT in english
\begin{resumo}[Abstract]
 \begin{otherlanguage*}{english}
  Knee osteoarthritis (OA) is one of the most common and disabling joint conditions worldwide. It is characterized as a progressive disease that primarily affects the knee cartilage. Although it has no cure, early detection is crucial to prevent its progression. Radiography is the main technique used for diagnosing OA and for classifying it based on the Kellgren/Lawrence (KL) scale. However, radiological diagnosis depends on the experience, interpretation, and time of the professional, which can lead to inconsistencies or errors. In this context, deep learning techniques offer a faster and more efficient alternative, enabling the automation of OA detection and classification.

  This study proposes a comparison between convolutional neural networks (CNNs) and vision transformers (ViTs) for the task of classifying knee OA severity, including models such as ResNet-34, ResNet-50, ResNet-101, VGG-16, VGG-19, DenseNet-121, DenseNet-169, Inception-v3, DeiT, Swin Transformer, DaViT, MaxViT, and GCViT. The models were trained using transfer learning, and the comparative analysis considers performance metrics, computational cost, quantitative uncertainty analysis, and interpretability. The results showed that CNN architectures, particularly those from the DenseNet family, achieved the best overall performance, with the DenseNet-169 model reaching an accuracy of 78.85\%. In terms of computational efficiency, CNNs were significantly faster, with DenseNet-121 offering the best balance between high predictive performance (QWK of 0.8878) and low training and inference cost (3.11 ms/image). Although competitive, ViTs showed lower performance and higher computational cost. Finally, the interpretability analysis using Grad-CAM confirmed that the top-performing models base their decisions on relevant pathological markers, such as joint space and osteophytes.

   \vspace{\onelineskip}
 
   \noindent 
   \textbf{Keywords}: Classification. convolutional neural networks. knee osteoarthritis. radiographs. transfer-learning. vision transformers.
 \end{otherlanguage*}
\end{resumo}

% Lista de ilustrações
\pdfbookmark[0]{\listfigurename}{lof}
\listoffigures*
\cleardoublepage

% Lista de tabelas
\pdfbookmark[0]{\listtablename}{lot}
\listoftables*
\cleardoublepage

% Lista de abreviaturas e siglas
\begin{siglas}
  \item[OA] Osteoartrite
  \item[KL] Kellgren/Lawrence
  \item[IA] Inteligência Artificial
  \item[RNC] Rede Neural Convolucional
  \item[ViT] Vision Transformer
  \item[WHO] World Health Organization
  \item[OAI] Osteoarthritis Initiative
  \item[NIH] National Institutes of Health
  \item[CAM] Class Activation Mapping
  \item[GAP] Global Average Pooling
\end{siglas}

% Lista de símbolos
% \begin{simbolos}
%   \item[$ \Gamma $] Letra grega Gama
%   \item[$ \Lambda $] Lambda
%   \item[$ \zeta $] Letra grega minúscula zeta
%   \item[$ \in $] Pertence
% \end{simbolos}

% Inserir o SUMÁRIO
\pdfbookmark[0]{\contentsname}{toc}
\tableofcontents*
\cleardoublepage

% ----------------------------------------------------------
% ELEMENTOS TEXTUAIS (Capítulos)
% ----------------------------------------------------------
\textual
% Elementos textuais com numeração arábica
\pagenumbering{arabic}
% Reinicia a contagem do número de páginas
\setcounter{page}{1}

% Inclui cada capitulo da Dissertação
% ----------------------------------------------------------
% Introdução 
% Capítulo sem numeração, mas presente no Sumário
% ----------------------------------------------------------

\chapter[Introdução]{Introdução}
% \addcontentsline{toc}{chapter}{Introdução}

A osteoartrite (OA), popularmente conhecida como artrose, é uma forma muito comum de doença reumática, caracterizada como uma condição multifatorial e degenerativa que afeta desde a cartilagem articular até os ossos adjacentes, resultando em sintomas de dor, deformidade e perda de função \cite{Kraus2015, PACCA2018}. Esses impactos comprometem significativamente a qualidade de vida, especialmente em grupos mais afetados, como idosos, mulheres e indivíduos obesos \cite{PACCA2018}. Além de sua alta prevalência, a OA é uma das principais causas de incapacidade no mundo, com maior incidência na articulação do joelho, seguida pelo quadril e pela mão. Dados de 2020 apontam que a doença afeta cerca de 7,6\% da população global, e projeções indicam um aumento de 60 a 100\% até 2050 \cite{COURTIES20241397}.

Exercícios de propriocepção e fortalecimento muscular, assim como terapias farmacêuticas, têm sido aplicados a pacientes diagnosticados com OA de joelho com o objetivo de controlar ou reduzir os sintomas de dor, uma vez que não existem medicamentos capazes de retardar o seu desenvolvimento \cite{Sardim2020, Lin2009}. Essa abordagem é especialmente apropriada para pacientes em estágios iniciais da doença, quando a cartilagem ainda não foi completamente degradada \cite{Kanamoto2020}. No entanto, o diagnóstico depende da experiência e do julgamento clínico do profissional na interpretação das radiografias, o que pode levar a inconsistências entre o grau previsto e o grau real, devido às mínimas diferenças entre os estágios adjacentes da doença \cite{KELLGREN1957, Mohammed2023}. Esses desafios têm impulsionado estudos sobre sistemas automáticos de detecção e classificação da OA de joelho.

A introdução de técnicas de inteligência artificial (IA) nos últimos anos tem permitido a automação de tarefas que antes eram realizadas manualmente, incluindo a interpretação de imagens médicas \cite{WANG2024103201}. Alguns exemplos incluem a detecção de pneumonia \citeonline{9077899}, a identificação e classificação de câncer de pulmão em tomografias computadorizadas e a detecção de retinopatia diabética em imagens de fundo de olho \cite{8697352, Dai2021}.

No campo da reumatologia, a visão computacional também tem sido aplicada à detecção de OA de joelho a partir de radiografias, com o objetivo de automatizar o processo de diagnóstico, reduzir a subjetividade da interpretação humana e realizar a classificação da severidade da doença através da escala de Kellgren/Lawrence (KL) \cite{Mohammed2023}. Esses estudos têm se concentrado em utilizar arquiteturas de aprendizado profundo, como redes neurais convolucionais (RNCs), e compará-las entre si para identificar qual abordagem oferece melhor desempenho na classificação da severidade da OA. No entanto, a operação de convolução limita o relacionamento entre pixels distantes em uma imagem, o que pode prejudicar a capacidade de captar dependências de longo alcance em radiografias \cite{Shamshad2023}.

Como uma abordagem alternativa, ou até complementar, foram propostas arquiteturas baseadas em transformers, capazes de apresentar um excelente desempenho em tarefas de classificação, como é o caso do vision transformer (ViT) \cite{Dosovitskiy2021}. Essas arquiteturas têm sido aplicadas com sucesso em tarefas relacionadas à medicina, como o diagnóstico de COVID-19 a partir de radiografias, classificação de tumores e doenças de retina, tornando-se o estado da arte nesta área \cite{Shamshad2023}.

Apesar dos avanços, persiste uma lacuna na literatura quanto a uma análise comparativa sistemática que avalie não apenas o desempenho preditivo, mas também a eficiência computacional e a interpretabilidade das RNCs em contraposição aos ViTs para a classificação ordinal da OA de joelho. Este trabalho, portanto, busca responder à seguinte questão de pesquisa: ``Qual família de arquiteturas, RNC ou ViT, oferece o melhor balanço entre acurácia, robustez ordinal, eficiência e interpretabilidade para a classificação da severidade da OA de joelho a partir de radiografias?''. Para guiar esta investigação de forma objetiva, o estudo testará as seguintes hipóteses:

\begin{itemize}
    \item \textbf{Hipótese 1 (desempenho base):} Modelos de RNC e ViT, quando treinados com a técnica de transferência de aprendizado, são capazes de classificar o grau de osteoartrite em radiografias de joelho com desempenho significativamente superior ao acaso, atingindo valores de F1-score macro superiores a 0,70 e de Quadratic Weighted Kappa (QWK) superiores a 0,80.

    \item \textbf{Hipótese 2 (comparação entre arquiteturas):} As RNCs, devido ao seu forte viés indutivo para o processamento de imagens, apresentarão desempenho preditivo (acurácia, QWK e F1-score macro) igual ou superior aos ViTs, com um custo computacional substancialmente menor, refletido em um tempo de inferência inferior a 50\% do observado nos ViTs de capacidade similar.
    
    \item \textbf{Hipótese 3 (consistência ordinal):} A utilização de uma função de perda ordinal (CORN), que reconhece a relação de ordem entre os graus de severidade, resultará em modelos com maior consistência clínica em comparação com a abordagem categórica padrão (Entropia Cruzada). Essa melhoria será quantificada por um aumento no QWK e uma redução no \textit{Mean Absolute Error} (MAE), mesmo que a acurácia geral não seja necessariamente superior.

    \item \textbf{Hipótese 4 (interpretabilidade clínica):} Os mapas de ativação (Grad-CAM) gerados pelos modelos de melhor desempenho destacarão predominantemente a região do espaço articular e as margens ósseas, áreas clinicamente relevantes para o diagnóstico da OA, demonstrando que o aprendizado não se baseia em características espúrias.
\end{itemize}

\section{Objetivos}

\subsection{Objetivo Geral}

O objetivo geral deste trabalho consiste em realizar uma análise comparativa completa entre modelos de RNC e ViT na tarefa de detectar e classificar a OA de joelho usando radiografias, com o intuito de validar as hipóteses propostas e identificar a abordagem mais adequada para uma potencial ferramenta de diagnóstico automatizado. E para que esse objetivo seja alcançado, foram definidos alguns objetivos específicos, conforme descrito a seguir.

\subsection{Objetivos Específicos}

\begin{itemize}
    \item Realizar uma revisão bibliográfica sobre a OA de joelho e as técnicas de visão computacional aplicadas à detecção de doenças reumáticas;
    \item Treinar os modelos propostos para classificar a severidade da OA de joelho a partir de um conjunto de dados público;
    \item Comparar os modelos de RNC e ViT com base em métricas de performance, eficiência computacional, incerteza preditiva e interpretabilidade;
    \item Analisar os resultados obtidos e discutir as vantagens e desvantagens de cada abordagem.
\end{itemize}

A metodologia proposta para atingir os objetivos deste trabalho consistiu nas seguintes etapas: coleta e pré-processamento de um conjunto de dados de radiografias de joelhos com diferentes graus de severidade da OA seguindo a escala KL; implementação da \textit{pipeline} de treinamento dos modelos para classificar a severidade da OA de joelho com hiperparâmetros fixos; avaliação dos modelos com base em métricas de performance, tempos de treinamento e inferência; aplicação da predição conformal para análise quantitativa; interpretação visual dos mapas de ativação; análise dos resultados obtidos e discussão das vantagens e desvantagens de cada abordagem.

\section{Organização do Trabalho}

Este trabalho está organizado em seis capítulos, incluindo a introdução. No \autoref{cap:fundamentacao}, são apresentados os conceitos e definições necessárias para o entendimento deste trabalho, incluindo a osteoartrite de joelho e suas características clínicas, conceitos fundamentais das arquiteturas de aprendizado profundo, incluindo as RNCs e os ViTs. No \autoref{cap:trab_relacionados}, são abordados os trabalhos relacionados. No \autoref{cap:proposta}, é apresentada a metodologia proposta para atingir os objetivos deste trabalho, assim como a avaliação dos modelos. No \autoref{cap:resultados}, são apresentados os resultados obtidos e discussões. Por fim, no \autoref{cap:conclusao}, são apresentadas as conclusões finais deste trabalho, apontando as contribuições, limitações e sugestões para trabalhos futuros.

\chapter{Fundamentação Teórica}\label{cap:fundamentacao}

Neste capítulo, são apresentados os conceitos e as definições necessárias para o entendimento deste trabalho. A Seção \ref{sec:osteoartrite} apresenta a osteoartrite de joelhos e suas características clínicas. A seção \ref{sec:visao-computacional} aborda a visão computacional na área da saúde. A Seção \ref{sec:aprendizado-profundo} mostra alguns conceitos fundamentais de arquiteturas de aprendizado profundo, incluindo as redes neurais convolucionais e os vision transformers.

% Por fim, a seção \ref{sec:ajuste-fino} apresenta os conceitos de transferência de aprendizado através do ajuste fino de modelos pré-treinados.

\section{Osteoartrite de Joelhos}\label{sec:osteoartrite}

A osteoartrite (OA) é uma doença heterogênea e degenerativa, que afeta as articulações e estruturas ósseas de pacientes \cite{Kraus2015}. A OA é a forma mais comum de doença articular, com uma prevalência global estimada em 365 milhões de indivíduos em 2023, e uma das principais causas de incapacidade no mundo, sendo altamente prevalente em idosos e indivíduos obesos \cite{Luis2022}. A OA é caracterizada por sintomas de dor, rigidez e mobilidade articular limitada, que podem comprometer significativamente a qualidade de vida dos pacientes. Embora possa afetar várias articulações, como ombros, cotovelos, pulso, coluna, entre outros, a OA é mais comum em joelhos e quadris \cite{PACCA2018}, onde a cartilagem articular é mais suscetível a desgastes causados pela carga do corpo.

A prevalência da OA cresceu 132\% nos últimos 30 anos, cuja projeção é de crescimento de 60 a 100\% até 2050. É observado também que a prevalência está correlacionada com o status socioeconômico do país, sendo mais comum em países desenvolvidos, como os Estados Unidos, onde quase 10\% da população adulta é afetada pela doença. Entre as causas da OA, estão fatores genéticos, idade, sexo, obesidade, trauma articular, entre outros. Entretanto, embora a OA seja uma doença multifatorial, a obesidade é um dos principais fatores de risco, contribuindo com aproximadamente 20\% no crescimento dos casos, uma vez que o excesso de peso aumenta a carga nas articulações, acelerando o desgaste da cartilagem (OA Prevalence and Burden) \cite{Courties2024}.

Kellgren e Lawrence \cite{Kanamoto2020} propuseram uma escala de classificação da OA baseada em radiografias e considerando fatores como a formação de osteófitos, estreitamento da cartilagem articular e esclerose subcondral. A escala de Kellgren/Lawrence (KL) classifica a OA em cinco estágios de progressão (\autoref{tabela-kl}): 0 (nenhum), 1 (duvidoso), 2 (mínimo), 3 (moderado) e 4 (grave) \cite{KELLGREN1957}. Tal classificação é comumente feita por radiologistas, que avaliam as radiografias e atribuem um grau de acordo com a experiência e cuidado médico na interpretação das imagens. No entanto, a classificação manual pode ser subjetiva e suscetível a erros, assim como foi observado pelos autores, o que pode levar a diagnósticos tardios ou incorretos num cenário onde a detecção precoce é crucial para retardar a progressão da doença, uma vez que não existem medicamentos capazes de retardar o seu desenvolvimento.

\begin{table}[htbp]
    \centering
    \begin{tabular}{|c|c|}
        \hline
        \textbf{Classe KL} & \textbf{Exemplo de Imagem} \\
        \hline
        0 (saudável) & \includegraphics[width=2cm]{figs/KL0-sample.png} \\
        \hline
        1 (duvidoso) & \includegraphics[width=2cm]{figs/KL1-sample.png} \\
        \hline
        2 (mínimo) & \includegraphics[width=2cm]{figs/KL2-sample.png} \\
        \hline
        3 (moderado) & \includegraphics[width=2cm]{figs/KL3-sample.png} \\
        \hline
        4 (severo) & \includegraphics[width=2cm]{figs/KL4-sample.png} \\
        \hline
    \end{tabular}
    \caption{Escala de Kellgren/Lawrence para classificação da severidade de osteoartrite.}
    \label{tabela-kl}
\end{table}

\section{Visão Computacional na Saúde}\label{sec:visao-computacional}

A visão computacional é uma subárea da inteligência artificial (IA) que tem como objetivo automatizar a análise de imagens digitais, permitindo que a máquina "veja" e interprete o conteúdo visual de uma imagem. Essa ideia emergiu por volta da década de 1960, quando pioneiros como David Marr e Hans Moravec se questionaram da possibilidade de tornar computadores capazes de enxergar. Desde então, com o desenvolvimento de pesquisas na área de IA e melhorias em hardware, houveram diversos avanços na área, como o surgimento de algoritmos para detecção de bordas, detecção de objetos, segmentação de imagens, entre outros \cite{huggingface2024}.

Na área da saúde, a visão computacional tem sido muito utilizada para melhorar a acurácia de diagnósticos, automatização de tarefas clínicas e tratamentos médicos. Ao analisar imagens médicas, como radiografias, tomografias, ressonâncias magnéticas e ultrassonografias, a máquina pode detectar e classificar patologias com maior precisão e rapidez do que um médico humano. Além disso, a visão computacional pode ser utilizada para monitorar o progresso de doenças, monitorar a eficácia de tratamentos e até mesmo recomendar tratamentos personalizados para pacientes \cite{JAVAID2024792}.

\section{Aprendizado Profundo}\label{sec:aprendizado-profundo}

O uso de modelos de aprendizado profundo baseados em redes neurais convolucionais (RNCs) tem ganhado espaço em tarefas de visão computacional. Aprendizado por transferência também é amplamente utilizado para reduzir o uso de recursos computacionais para tarefas que já são executadas por modelos existentes, como as redes residuais (ResNet), Visual Geometry Group (VGG) e as redes densamente conectadas (DenseNet) \cite{Tariq2023}. Enquanto o uso de RNCs tem se mostrado útil em soluções de detecção em imagens médicas, a operação de convolução limita o relacionamento entre pixels distantes numa imagem. Para tanto, a habilidade de codificar dependências de longo alcance tem sido possível graças às arquiteturas de aprendizado profundas baseadas em atenção, como o Vision Transformer (ViT). Tais modelos de ViT têm sido empregados para várias tarefas, incluindo classificação e detecção de objetos \cite{Shamshad2023}.

% \section{Transferência de Aprendizado e Ajuste Fino}\label{sec:ajuste-fino}

% PARTE - Define a divisão do documento em partes (Não é obrigatório)
% \part{Preparação da pesquisa}
% \include{capitulos/referencias}
\chapter{Trabalhos Relacionados}\label{cap:trab_relacionados}

A OA de joelho é uma área de pesquisa ativa na medicina e na ciência da computação, especialmente com o advento de técnicas de visão computacional. Este capítulo revisa alguns trabalhos relevantes que abordam a detecção e classificação da doença, destacando as metodologias e resultados obtidos.

Em 2023, \citeonline{Tariq2023} apresentaram uma abordagem de classificação ordinal (5 classes) baseada em aprendizado profundo utilizando radiografias posteroanteriores de joelhos. O estudo aplicou a estratégia de aprendizado por transferência ao fazer o ajuste fino de modelos pré-treinados, como ResNet-34, VGG-19, DenseNet-121 e DenseNet-169, combinando suas saídas em um modelo de \textit{ensemble} (\autoref{fig:tariq2023}). Usando o CORN como a função de perda, os autores alcançaram uma acurácia geral de 98\% e 0,99 de QWK.

\begin{figure}[!htbp]
    \centering
    \includegraphics[width=0.8\textwidth]{figs/tariq2023.png}
    \caption{Metodologia proposta por \citeonline{Tariq2023}.}
    \label{fig:tariq2023}
\end{figure}

Ainda em 2023, \citeonline{Mohammed2023} utilizaram seis modelos pré-treinados de RNC (VGG-16, VGG-19, ResNet-101, MobileNetV2, InceptionResNetV2 e DenseNet-121) para diagnosticar a OA de joelho, considerando vários cenários de teste, como a classificação binária e o nível de severidade com três e cinco classes (\autoref{fig:mohammed2023}). O destaque do estudo foi a experimentação dos modelos em diferentes cenários, modelando tanto a detecção, quanto a própria classificação da OA, através do agrupamento das radiografias. O modelo ResNet-101 registrou as acurácias máximas com cinco, duas e três classes, sendo 69\%, 83\% e 89\%, respectivamente.

\begin{figure}[!htbp]
    \centering
    \includegraphics[width=\textwidth]{figs/mohammed2023.jpg}
    \caption{Metodologia proposta por \citeonline{Mohammed2023}.}
    \label{fig:mohammed2023}
\end{figure}

Partindo para a introdução de um modelo customizado, os autores brasileiros \citeonline{domingues2023} propuseram um modelo de RNC baseado na arquitetura DenseNet-161, treinado com um conjunto de radiografias obtidas do Estudo Longitudinal de Saúde do Adulto Musculoesquelético (ELSA-Brasil Musculoesquelético), para a classificação binária automática da OA de joelho (\autoref{fig:domingues2023}). Eles aplicaram diversas técnicas de pré-processamento, como rotação, desfoque gaussiano e inversão horizontal, e alcançaram uma AUC de 0,866 (IC 95\%: 0,842-0,882), considerando uma média entre os subconjuntos de treino e teste. O modelo também pode ser calibrado por meio do ajuste de limiares para alcançar uma acurácia máxima de 90,7\% e uma sensibilidade de 93,8\%.

\begin{figure}[!htbp]
    \centering
    \includegraphics[width=\textwidth]{figs/domingues2023.png}
    \caption{Metodologia proposta por \citeonline{domingues2023}.}
    \label{fig:domingues2023}
\end{figure}

\citeonline{Cueva2022} desenvolveram um sistema de diagnóstico por computação assistida (CAD) utilizando a técnica de ajuste fino do modelo ResNet-34 para detectar OA nos dois joelhos simultaneamente (\autoref{fig:cueva2022}). Os autores resolveram o problema de desequilíbrio do conjunto de dados por meio de técnicas de \textit{oversampling} e \textit{data augmentation}, como rotação aleatória e variação de cor. O modelo alcançou uma acurácia média de 61,71\% em múltiplas classes, com melhor desempenho para as classes KL-0, KL-3 e KL-4 em comparação com KL-1 e KL-2 devido às sutis diferenças nos estágios intermediários.

\begin{figure}[!htbp]
    \centering
    \includegraphics[width=\textwidth]{figs/cueva2022.png}
    \caption{Metodologia proposta por \citeonline{Cueva2022}.}
    \label{fig:cueva2022}
\end{figure}

Utilizando uma outra abordagem, \citeonline{yeoh2023} investigaram o uso de redes neurais convolucionais 3D para a detecção binária de OA de joelho a partir de imagens de ressonância magnética 3D. O estudo também utilizou transferência de aprendizado, transformando pesos de modelos pré-treinados em 2D para 3D. A abordagem permitiu capturar informações espaciais nas três dimensões, resultando em uma acurácia de 87,5\% e um F1-score de 0,871 para o melhor modelo, o ResNet-34.

Com a introdução dos ViTs, novas possibilidades surgiram para trabalhar o mesmo problema, oferecendo uma alternativa às RNCs, por vezes superando-as em tarefas de classificação de imagens. Em 2023, \citeonline{sekhri2023} introduziram uma abordagem utilizando o Swin Transformer para previsão da severidade da OA de joelho. Para lidar com a alta similaridade entre os graus adjacentes da escala KL, eles implementaram uma arquitetura de múltiplas previsões composta por cinco redes perceptron multicamadas (MLP), cada uma dedicada a prever um grau específico de KL. Além disso, para reduzir o desvio de dados entre os conjuntos de dados (OAI e MOST), congelaram as camadas MLP após o treinamento inicial em um conjunto de dados e continuaram treinando o extrator de características em outro para alinhar os espaços representacionais latentes. Essa abordagem alcançou acurácia de 70,17\% e F1-score de 0,67 no conjunto de dados OAI, superando os métodos existentes do estado da arte.

\begin{figure}[!htbp]
    \centering
    \includegraphics[width=0.7\textwidth]{figs/sekhri2023.png}
    \caption{Metodologia proposta por \citeonline{sekhri2023}.}
    \label{fig:sekhri2023}
\end{figure}

\citeonline{Wang_2024} criaram um modelo baseado em ViT para a detecção precoce da OA de joelho, focando na distinção entre o grau KL-0 e KL-2 (\autoref{fig:wang2024}). A metodologia incorporou três inovações principais:

\begin{itemize}
    \item \textit{Selective Shuffled Position Embedding} (SSPE): Ao fixar o posicionamento de ``patches-chave'' (regiões com características de grau KL) e embaralhar os demais, o modelo foi forçado a focar nas áreas críticas afetadas pela OA.
    \item Estratégia de troca de patches-chave: Como a técnica de aumento de dados, patches-chave de imagens candidatas foram trocados com a imagem alvo para gerar sequências de entrada diversas.
    \item Função de perda híbrida: Uma combinação de \textit{Label Smoothing Cross-Entropy} (LSCE) para sequências mistas de grau KL e \textit{cross-entropy} (CE) para sequências completas de grau KL foi otimizada para melhorar a generalização do modelo.
\end{itemize}

Essas estratégias resultaram em uma melhoria notável no desempenho de classificação, com o modelo alcançando uma acurácia de 89,80\%.

\begin{figure}[!htbp]
    \centering
    \includegraphics[width=\textwidth]{figs/wang2024.png}
    \caption{Metodologia proposta por \citeonline{Wang_2024}.}
    \label{fig:wang2024}
\end{figure}

Seguindo uma linha semelhante a este estudo, \citeonline{apon2024} conduziram uma análise comparativa entre modelos ViT pré-existentes (DaViT, GCViT, MaxViT) e RNCs tradicionais (\autoref{fig:apon2024}). Eles destacaram as forças arquitetônicas do DaViT com auto-atenção dupla, do GCViT com auto-atenção de contexto global e do MaxViT com atenção multi-eixo. Esses modelos ViT se destacaram com as melhores métricas, alcançando uma acurácia máxima de 66,14\%, precisão de 0,703, revocação de 0,614 e AUC superior a 0,835, superando consistentemente as RNCs (com acurácia entre 55-65\%).

\begin{figure}[!htbp]
    \centering
    \includegraphics[width=\textwidth]{figs/apon2024.png}
    \caption{Metodologia proposta por \citeonline{apon2024}.}
    \label{fig:apon2024}
\end{figure}
% \include{capitulos/ferramentas}

% PARTE
% \part{Proposta}
\chapter{Metodologia}\label{cap:proposta}

Esta seção descreve a metodologia proposta para a tarefa de classificação da OA de joelho a partir de radiografias. A principal abordagem desta pesquisa consiste no uso de \textit{transfer learning} para aproveitar o conhecimento já obtido por modelos pré-treinados e melhorar a performance da predição final.

\section{Coleta de dados}

A seleção e coleta de dados constituem etapas iniciais fundamentais no desenvolvimento de modelos de aprendizado profundo. Nesse estudo, o conjunto de dados (ou \textit{dataset} do inglês) foi obtido por meio da plataforma Kaggle \citep{dataset-kaggle}, amplamente reconhecida por disponibilizar dados de alta qualidade e de acesso público para fins acadêmicos. O \textit{dataset} escolhido baseia-se na Osteoarthritis Initiative (OAI) e contém 9.786 radiografias de joelho rotuladas com suas respectivas classificações de severidade da OA, seguindo o sistema de Kellgren-Lawrence (\autoref{tabela-kl}). A escolha desta fonte deve-se à sua ampla utilização na plataforma e na literatura \citep{Tariq2023, Mohammed2023}, além do volume de imagens, fornecendo uma base sólida e representativa para o treinamento e avaliação dos modelos propostos. Um resumo do \textit{dataset} é apresentado na \autoref{dataset-summary}.

\begin{table}[ht]
    \centering
    \begin{tabular}{|c|c|c|c|}
        \hline
        \textbf{Classe KL} & \textbf{Descrição} & \textbf{Total de imagens} & \textbf{\% do total} \\
        \hline
        0 & saudável & 3857 & 40\% \\
        1 & duvidoso & 1770 & 18\% \\
        2 & mínimo & 2578 & 26\% \\
        3 & moderado & 1286 & 13\% \\
        4 & severo & 295 & 3\% \\
        \hline
        \textbf{Total} & - & 9786 & 100\% \\
        \hline
    \end{tabular}
    \caption{Número de radiografias por classe KL no conjunto de dados original.}
    \label{dataset-summary}
\end{table}

Todas as imagens possuem resolução de 224x224 pixels e estão no formato PNG. As imagens foram agrupadas em subconjuntos de treino, teste, validação e calibração, com uma proporção de 7:1:1:1. O conjunto de treino é utilizado para treinar os modelos, o conjunto de validação é usado para ajustar os hiperparâmetros e monitorar o desempenho do modelo durante o treinamento, o conjunto de teste é utilizado para avaliar o desempenho final do modelo e verificar sua capacidade de generalização em dados novos, e o conjunto de calibração é usado para aplicar a estratégia de predição conformal, discutida na \autoref{sec:conformal-prediction}. A distribuição das imagens por subconjunto de dados pode ser visualizada na \autoref{dataset-distribuition}.

\begin{figure}[ht]
    \centering
    \includegraphics[width=0.7\linewidth]{figs/dataset-class-distribution.png}
    \caption{Distribuição das radiografias por classe KL nos subconjuntos de treino, teste, validação e calibração.}
    \label{dataset-distribuition}
\end{figure}

Com o objetivo de explorar diferentes abordagens para a classificação da severidade da OA de joelho, foram derivados, a partir do \textit{dataset} original contendo cinco classes, três novos conjuntos de dados: com 4, 3 e 2 classes. O conjunto com 4 classes foi construído por meio da exclusão da classe 1 (duvidosa), com a finalidade de simplificar o problema de classificação. O conjunto com 3 classes foi obtido pela remoção das classes 0 e 1 (respectivamente, saudável e duvidosa), resultando em um subconjunto composto apenas pelas instâncias que apresentavam algum grau de severidade (mínima, moderada ou severa). Por fim, o conjunto com 2 classes foi gerado ao se agrupar as classes 0 e 1, representando a ausência de OA, e as classes 2, 3 e 4, representando a presença de OA, formando, assim, um conjunto de dados binário.

\section{Pré-processamento das imagens}

A etapa de pré-processamento é essencial para garantir que as imagens estejam em um formato adequado para o treinamento dos modelos. Neste estudo, o pré-processamento das radiografias foi dividido em duas etapas: pré-processamento geral e pré-processamento específico para cada modelo. O pré-processamento geral, realizado antes do treinamento, inclui técnicas como equalização de histograma e filtro gaussiano. Já o pré-processamento específico para cada modelo, realizado durante o treinamento, envolve a adaptação das imagens às exigências de entrada dos modelos selecionados, como redimensionamento e normalização dos valores dos pixels. Além disso, o aumento de dados foi aplicado para expandir a variabilidade do conjunto de dados e mitigar o efeito do desbalanceamento entre as classes.

\subsection{Equalização de Histograma}

A equalização de histograma foi utilizada como técnica de pré-processamento com o intuito de melhorar o contraste das radiografias coletadas do conjunto original. Esse método redistribuiu os níveis de intensidade dos pixels de forma a abranger a maior faixa de valores possíveis, aumentando a separabilidade entre as regiões mais claras e mais escuras da radiografia. Em particular, essa técnica foi útil para realçar o contraste das estruturas ósseas e o espaço articular do joelho, assim como alterações ósseas sutis que podem ser indicativas de OA.

A aplicação da equalização de histograma foi realizada utilizando a biblioteca OpenCV \citep{opencv} do Python. A \autoref{fig:histogram-equalization}(a) ilustra uma radiografia original do joelho, enquanto a \autoref{fig:histogram-equalization}(b) mostra a mesma radiografia após a equalização de histograma. É possível observar que a equalização melhorou o contraste da imagem, tornando as estruturas ósseas mais visíveis. As respectivas distribuições de intensidade dos pixels antes e depois da equalização são apresentadas na \autoref{fig:histogram-equalization-histogram}.

\begin{figure}
    \centering
    \begin{tabular}{@{}c@{}}
        \includegraphics[width=0.45\textwidth]{figs/imagem-nao-equalizada.png} \\[\abovecaptionskip]
        \small (a) Radiografia original do joelho.
    \end{tabular}
    \hfill
    \begin{tabular}{@{}c@{}}
        \includegraphics[width=0.45\textwidth]{figs/image-equalizada.png} \\[\abovecaptionskip]
        \small (b) Radiografia após equalização de histograma.
    \end{tabular}
    \caption{Exemplo de equalização de histograma aplicada a uma radiografia de joelho.}
    \label{fig:histogram-equalization}
\end{figure}

\begin{figure}
    \centering
    \begin{tabular}{@{}c@{}}
        \includegraphics[width=0.45\textwidth]{figs/histograma-imagem-nao-equalizada.png} \\[\abovecaptionskip]
        \small (a) Histograma da radiografia original.
    \end{tabular}
    \hfill
    \begin{tabular}{@{}c@{}}
        \includegraphics[width=0.45\textwidth]{figs/histograma-imagem-equalizada.png} \\[\abovecaptionskip]
        \small (b) Histograma da radiografia após equalização.
    \end{tabular}
    \caption{Distribuições de intensidade dos pixels antes e depois da equalização de histograma.}
    \label{fig:histogram-equalization-histogram}
\end{figure}

\subsection{Normalização}

A normalização das radiografias consistiu em uma etapa fundamental do pré-processamento, com o objetivo de padronizar a escala dos valores dos pixels e, assim, facilitar o aprendizado pelos modelos. Essa técnica foi aplicada convertendo os valores de intensidade dos pixels, originalmente na faixa de 0 a 255, para uma faixa padronizada entre 0 e 1.

Neste estudo, a normalização foi implementada em todos os subconjuntos de dados utilizando a função \texttt{transforms.Normalize} da biblioteca PyTorch \citep{pytorch}, que aplica a normalização em cada canal (RGB), subtraindo a média e dividindo pelo desvio padrão. Para modelos baseados em arquiteturas tradicionais, como ResNet e VGG, utilizaram-se os valores convencionais:

\begin{itemize}
    \item Média: 0.485, 0.456 e 0.406
    \item Desvio padrão: 0.229, 0.224 e 0.225
\end{itemize}

Para modelos baseados em ViTs, como o DeiT e o Swin Transformer, foram utilizados os valores de normalização específicos para esses modelos, obtidos diretamente do objeto \texttt{processor}, utilizando a função \texttt{processor.image\_mean} e \texttt{processor.image\_std}, garantindo a compatibilidade com o pré-processamento original desses modelos.

\subsection{Aumento de dados}

Com o objetivo de melhorar a generalização dos modelos e reduzir o risco de \textit{overfitting}, foi aplicado o aumento de dados (\textit{data augmentation}) nas radiografias durante o treinamento dos modelos.

A técnica consistiu na aplicação de transformações geométricas simples nas imagens do conjunto de treinamento, de forma a simular variações naturais que poderiam ocorrer nas radiografias. As transformações incluíram a inversão horizontal (reflexão), com probabilidade de 50\%, e rotações aleatórias limitadas a um intervalo de -10 a 10 graus.

Antes das transformações, as imagens foram redimensionadas para o tamanho esperado pelo modelo, definido como 224x224 pixels para todos os modelos, exceto para o modelo InceptionV3, que requer imagens de 299x299 pixels.

\subsection{Subamostragem}

Como pode ser observado na \autoref{dataset-summary}, o conjunto de dados original apresenta um desbalanceamento significativo entre as classes, com a classe 0 (saudável) representando 40\% do total de imagens e a classe 4 (severo) apenas 3\%. Para lidar com esse desbalanceamento, além do aumento de dados, foi aplicada a técnica de subamostragem (\textit{undersampling}) nas classes majoritárias e reduzindo o número de imagens dessas classes, equilibrando sua proporção em relação às classes minoritárias.

A subamostragem foi aplicada apenas no conjunto de treinamento, de modo a não comprometer a representatividade das distribuições no conjunto de validação, testes e calibração. A técnica consistiu na seleção aleatória de um subconjunto das amostras das classes até um limite definido de 1.700 imagens por classe. Esse limite foi escolhido com base na classe 2 (mínima), que possui o maior número de imagens entre as classes com severidade, garantindo que todas as classes fossem representadas de forma equilibrada no conjunto de treinamento.

Embora essa estratégia possa levar à perda de informações potencialmente úteis, ela ajuda a reduzir o viés do modelo em direção às classes majoritárias e melhora sua capacidade de aprender padrões relevantes em todas as classes.

% PARTE
% \part{Parte Final}
\chapter{Resultados esperados}\label{cap:resultados}

Esta pesquisa utiliza transfer learning de modelos pré-treinados e faz um ajuste fino com o objetivo de classificar o nível de severidade da osteoartrite de joelho usando a escala de Kellgren/Lawrence. Para isso, os treinamentos serão feitos usando a linguagem de programação Python em um ambiente de notebooks disponíveis na plataforma do Google Colab. Caso o treinamento exija um maior poder computacional, o super computador da Universidade Federal do ABC poderá ser utilizado para esta pesquisa.

Nesta seção será discutido os resultados esperados quanto à performance dos modelos treinados usando a metodologia proposta. Espera-se que existam variações significativas de performance entre os modelos, dado que cada arquitetura possui diferentes estratégias para a extração e processamento das características contidas nas radiografias de joelho.

Os modelos ResNet (ResNet34, ResNet50 e ResNet101) são eficazes para extrair características complexas em imagens, incluindo imagens médicas como radiografias de joelho. Sabe-se que quanto mais profunda é a rede, maior é sua capacidade de capturar padrões complexos. Nesse sentido, o modelo ResNet101 (101 camadas) tende a ser o modelo com maior acurácia, mas pode sofrer com \textit{overfitting} se o conjunto de dados for pequeno, como acontece para a classe KL 4, com apenas 295 imagens no total. Além disso, é esperado que o consumo de recursos computacionais seja maior para redes mais profundas. O modelo ResNet50 pode oferecer um bom equilíbrio entre generalização do modelo e custo computacional.

Os modelos VGG (VGG16 e VGG19), apesar de profundos, possuem uma arquitetura mais simples em comparação com o ResNet e são menos eficientes em termos de uso de parâmetros e, portanto, identificação de características complexas das radiografias. Embora seja esperado que estes modelos apresentem performance inferior em relação aos modelos ResNet considerando suas profundidades, pelo fato deles serem modelos com arquitetura mais simples e direta, consistindo principalmente de camadas convolucionais empilhadas seguidas por camadas totalmente conectadas, isso pode levar a uma melhor generalização do modelo e menor probabilidade de \textit{overfitting}, trazendo uma performance melhor.

Os modelos DenseNet (DenseNet121 e DenseNet169) possuem conexões diretas entre todas as camadas, o que facilita o fluxo de informação e melhora a eficiência do aprendizado de padrões em imagens. Esses modelos podem ser especialmente úteis em capturar detalhes sutis nas radiografias, como pequenas degradações no espaço articular. Logo, espera-se que estes modelos tenham uma performance muito competitiva e superem os modelos ResNet e VGG, principalmente o DenseNet169.

O GoogLeNet, com sua arquitetura Inception, permite que o modelo capture diferentes tamanhos de características simultaneamente. Essa flexibilidade pode ser benéfica em radiografias ao extrair padrões importantes em diferentes imagens ou até conjuntos de dados. É esperado que este modelo tenha um bom desempenho, mas não supere as arquiteturas anteriores.

Com relação aos modelos de transformer, o Vit-B/16 tem a capacidade de extrair informações globais da imagem usando um mecanismo de atenção sem a necessidade de convolução. Para o conjunto de dados desta pesquisa, espera-se que ele se destaque, capturando relações complexas e interdependências entre diferentes regiões da imagem, o que pode ser bom na classificação da osteoartrite de joelho. Entretanto, para a classe KL 4, pode ser que o modelo não tenha um desempenho tão bom, dado que os ViTs dependem de muitos dados para treinar efetivamente.

O DeiT é uma abordagem mais eficiente em termos de dados de treinamento, pois ele foi projetado para ser robusto em conjuntos de dados menores. Espera-se que ele tenha um bom equilíbrio entre performance e eficiência computacional, além de apresentar resultados competitivos em relação aos modelos RNCs.

O Swin Transformer é projetado para capturar características locais e globais através de uma abordagem hierárquica de atenção, o que o torna uma boa opção para tarefas que exigem análise de características em várias escalas, como em radiografias. Devido à sua capacidade de trabalhar em diferentes níveis de granularidade, é esperado que o Swin Transformer tenha um desempenho muito competitivo, podendo até mesmo ser superior aos outros modelos de ViT e RNC.

Por fim, o ResNet50-ViT-B/16 junta as forças de RNCs e ViTs para criar uma arquitetura promissora na tarefa de classificação da OA de joelho. A combinação de extração de características locais detalhadas pelo modelo ResNet50 com a capacidade dos transformers de capturar dependências globais na imagem oferece um equilíbrio vantajoso entre precisão e generalização. Espera-se que este modelo apresente resultados superiores em comparação com modelos puramente convolucionais e modelos de transformers isolados.
% \chapter{Conclusão}\label{cap:conclusao}

Este trabalho se propôs a realizar uma análise comparativa abrangente entre arquiteturas de redes neurais convolucionais (RNCs) e vision transformers (ViTs) para a tarefa de classificação da severidade da osteoartrite (OA) de joelho, utilizando a escala de Kellgren/Lawrence. Diante da subjetividade e do tempo demandado pelo diagnóstico manual, o objetivo central foi identificar os modelos mais robustos, eficientes e confiáveis, empregando uma avaliação com diferentes variáveis que incluiu desempenho preditivo, custo computacional, quantificação de incerteza e interpretabilidade.

A investigação sistemática das treze arquiteturas revelou conclusões claras e significativas. Os modelos da família DenseNet se destacaram como os de melhor desempenho geral. O DenseNet-169 alcançou a maior acurácia (78,85\%), enquanto o DenseNet-121 obteve o maior QWK (0,8878), demonstrando a melhor performance preditiva na tarefa de classificação ordinal.

A comparação entre as funções de perda de entropia cruzada e CORN evidenciou uma compensação fundamental. Enquanto a entropia cruzada maximizou a acurácia, a função CORN consistentemente melhorou o QWK, minimizando a gravidade dos erros de classificação. Adicionalmente, a análise com a predição conformal mostrou que a abordagem com CORN gera conjuntos de predição drasticamente mais informativos e úteis para a prática clínica.

A análise da eficiência computacional apresentou uma expressiva vantagem de eficiência das RNCs sobre os ViTs. Modelos como DenseNet-121 e ResNet-50 foram de quatro a cinco vezes mais rápidos em inferência do que os ViTs de alto desempenho, como DaViT-B e Swin-B. Esse resultado sublinha um desafio prático para a implantação de ViTs em ambientes clínicos que demandam baixo custo e alta velocidade.

A análise com Grad-CAM confirmou que os modelos de melhor desempenho basearam suas decisões em marcadores patológicos clinicamente relevantes, como o espaço articular e osteófitos. Também foram identificadas ``estratégias visuais'' distintas, com as RNCs produzindo ativações mais focadas e os ViTs, mais contextuais.

Apesar do rigor metodológico, este trabalho possui algumas limitações que devem ser reconhecidas. O estudo foi conduzido num único conjunto de dados público com um desbalanceamento claro. Além disso, a anaĺise se baseou apenas em imagens estáticas, sem qualquer informação extra sobre os pacientes. Trabalhos futuros podem explorar outros conjuntos de dados para garantir a generalização dos resultados, integrar com um estudo longitudinal que acompanhasse a progressão da OA ao longo do tempo, além de otimizar as arquiteturas com estratégias de \textit{grid-searching} e \textit{ensemble}, combinando a saída dos melhores modelos para determinar a classe KL.

Este trabalho demonstrou com sucesso que modelos de aprendizado profundo, em particular as RNCs e ViTs, são capazes de classificar a severidade da OA de joelho com alta acurácia e confiabilidade. Mais importante, o estudo evidenciou que uma avaliação completa deve ir além da acurácia, considerando a relevância clínica do erro, a eficiência computacional e a interpretabilidade. Assim, o estudo fornece um \textit{benchmark} sólido para o desenvolvimento de futuras ferramentas de IA que possam auxiliar profissionais da saúde no diagnóstico e manejo da OA de joelho.

% ----------------------------------------------------------
% ELEMENTOS PÓS-TEXTUAIS (Referências, Glossário, Apêndices)
% ----------------------------------------------------------
\postextual

% Referências bibliográficas
\bibliography{bibliografia}

% Glossário (Consulte o manual)
%\glossary

% Apêndices
% % ----------------------------------------------------------
% Apêndices
% ----------------------------------------------------------

% ---
% Inicia os apêndices
% ---
\begin{apendicesenv}

% Imprime uma página indicando o início dos apêndices
\partapendices

% ----------------------------------------------------------
\chapter{Resultados Suplementares}
% ----------------------------------------------------------

Como complemento à análise principal, foram conduzidos experimentos para avaliar o desempenho dos modelos na tarefa de classificar a OA de joelho em 4, 3 e 2 classes, através da manipulação das imagens do conjunto de dados.

\section{Classificação em 4 Classes}
\label{apendice:resultados_4_classes}

Inicialmente, foi realizado um experimento para a tarefa de classificação em 4 classes, excluindo a classe KL-1 (``duvidoso''). O objetivo foi investigar o impacto da remoção desta classe, identificada como ambígua no desempenho geral das arquiteturas.

A exclusão da classe KL-1 resultou em um aumento substancial e generalizado no desempenho de todos os modelos, validando a hipótese de que a ambiguidade desta classe era um dos principais desafios para a tarefa de classificação. A \autoref{tab:overall_metrics_4_classes} resume as métricas de desempenho para os modelos mais relevantes neste cenário.

O resultado mais notável foi o aumento significativo na performance. O modelo GCViT-B, treinado com a função de perda CORN, alcançou uma acurácia de 90,08\%, um QWK de 0,9311 e um MAE de 0,1635, estabelecendo-se como o modelo de melhor desempenho neste cenário. Outros modelos, como o DaViT-B, também apresentaram resultados expressivos, com 89,28\% de acurácia e 0,9267 de QWK.

Diferentemente do cenário de 5 classes, onde os modelos da família DenseNet se destacaram, a remoção da classe ambígua permitiu que os ViTs, em particular o GCViT-B e o DaViT-B, demonstrassem seu pleno potencial, superando as RNCs em acurácia e QWK.

\begin{table}[!htbp]
    \centering
    \caption{Resumo das métricas gerais de desempenho para modelos selecionados no cenário de 4 classes.}
    \label{tab:overall_metrics_4_classes}
    \begin{tabular}{|l|l|c|c|c|}
        \hline
        \textbf{Modelo} & \textbf{Função de perda} & \textbf{Acurácia} & \textbf{QWK} & \textbf{MAE} \\
        \hline
        ResNet-50 & \text{CORN} & 87,67\% & 0,9146 & 0,2105 \\
        \hline
        DenseNet-121 & \text{CORN} & 88,20\% & 0,9153 & 0,1957 \\
        \hline
        DenseNet-169 & \text{CORN} & 88,34\% & 0,9192 & 0,2051 \\
        \hline
        Inception-v3 & \text{Entropia Cruzada} & 88,61\% & 0,9225 & 0,1917 \\
        \hline
        DaViT-B & \text{Entropia Cruzada} & 89,28\% & 0,9267 & 0,1783 \\
        \hline
        GCViT-B & \text{CORN} & \textbf{90,08\%} & \textbf{0,9311} & \textbf{0,1635} \\
        \hline
    \end{tabular}
\end{table}

\section{Classificação em 3 Classes}
\label{apendice:resultados_3_classes}

Um segundo experimento foi conduzido para avaliar a capacidade dos modelos em distinguir exclusivamente entre os diferentes estágios de severidade da OA, uma vez que a doença já está estabelecida. Para este fim, as classes KL-0 (saudável) e KL-1 (duvidoso) foram removidas, resultando em um problema de classificação com 3 classes: KL-2 (mínimo), KL-3 (moderado) e KL-4 (severo).

Essa simplificação do problema, focando apenas nos estágios da doença, resultou em um desempenho preditivo alto em todas as arquiteturas avaliadas. A \autoref{tab:overall_metrics_3_classes} resume as métricas de desempenho para os modelos mais relevantes neste cenário. O resultado mais interessante foi a alta acurácia e concordância ordinal alcançadas. O modelo Inception-v3, treinado com a entropia cruzada,, atingiu a maior acurácia de 94,43\% e o maior QWK de 0,9285. Ainda assim, os demais modelos tiveram uma ótima performance.

Estes resultados indicam que os modelos avaliados possuem uma alta capacidade não apenas para detectar a OA, mas também para diferenciar seus estágios de severidade, uma tarefa fundamental para o planejamento de tratamentos e o acompanhamento da progressão da doença.

\begin{table}[!htbp]
    \centering
    \caption{Resumo das métricas gerais de desempenho para modelos selecionados no cenário de 3 classes.}
    \label{tab:overall_metrics_3_classes}
    \begin{tabular}{|l|l|c|c|c|}
        \hline
        \textbf{Modelo} & \textbf{Função de perda} & \textbf{Acurácia} & \textbf{QWK} & \textbf{MAE} \\
        \hline
        ResNet-101 & \text{Entropia Cruzada} & 92,91\% & 0,9082 & 0,0709 \\
        \hline
        DenseNet-121 & \text{CORN} & 93,42\% & 0,9136 & 0,0658 \\
        \hline
        DaViT-B & \text{Entropia Cruzada} & 93,92\% & 0,9218 & 0,0608 \\
        \hline
        DenseNet-169 & \text{CORN} & 93,92\% & 0,9220 & 0,0608 \\
        \hline
        Inception-v3 & \text{Entropia Cruzada} & \textbf{94,43\%} & \textbf{0,9285} &  \textbf{0,0557} \\
        \hline
        GCViT-B & \text{Entropia Cruzada} & 93,16\% & 0,9110 & 0,0684 \\
        \hline
    \end{tabular}
\end{table}

\section{Classificação em 2 Classes}
\label{apendice:resultados_2_classes}

Finalmente, um terceiro experimento foi realizado para avaliar a capacidade dos modelos em uma tarefa de detecção binária, que possui grande relevância clínica para a triagem inicial de pacientes. Neste cenário, as classes KL-0 e KL-1 foram agrupadas para representar a ``ausência de OA'', enquanto as classes KL-2, KL-3 e KL-4 foram consolidadas para representar a ``presença de OA''.

Os resultados, resumidos na \autoref{tab:overall_metrics_2_classes}, indicam que todos os modelos foram capazes de realizar a detecção binária com um desempenho robusto, alcançando acurácias na faixa de 84\% a 87\%. Nesta formulação do problema, os ViTs demonstraram uma ligeira vantagem. O modelo GCViT-B (entropia cruzada) destacou-se com a maior acurácia, atingindo 87,10\%, além do maior QWK de 0,7374 e MAE de 0,1290, indicando a melhor concordância ajustada ao acaso. Logo em seguida, o MaxViT-T (CORN) também apresentou um excelente resultado, com 86,77\% de acurácia.

As diferenças de desempenho entre as funções de perda foram mínimas e inconsistentes neste cenário, o que é esperado, uma vez que para um problema de duas classes a formulação ordinal do CORN se aproxima da logística binária padrão.

\begin{table}[!htbp]
    \centering
    \caption{Resumo das métricas gerais de desempenho para modelos selecionados no cenário de classificação binária.}
    \label{tab:overall_metrics_2_classes}
    \begin{tabular}{|l|l|c|c|c|}
        \hline
        \textbf{Modelo} & \textbf{Função de perda} & \textbf{Acurácia} & \textbf{QWK} & \textbf{MAE} \\
        \hline
        ResNet-50 & \text{CORN} & 84,90\% & 0,6956 & 0,1510 \\
        \hline
        DenseNet-121 & \text{CORN} & 85,67\% & 0,6861 & 0,1433 \\
        \hline
        DenseNet-169 & \text{CORN} & 85,56\% & 0,7055 & 0,1444 \\
        \hline
        DaViT-B & \text{CORN} & 86,55\% & 0,7268 & 0,1345 \\
        \hline
        MaxViT-T & \text{CORN} & 86,77\% & 0,7300 & 0,1323 \\
        \hline
        GCViT-B & \text{Entropia Cruzada} & \textbf{87,10\%} & \textbf{0,7374} & \textbf{0,1290} \\
        \hline
    \end{tabular}
\end{table}

\end{apendicesenv}
% ---

% Anexos
% \include{postextual/anexos}

% Índice remissivo (Consultar manual)
%\phantompart
%\printindex

\end{document}
